\chapter{Protein Models}
\label{chap:protein}


\section{Clementi G\={o}}
\label{sec:protein_cc_go}

\todo{} Brief description and references.

\subsection{Topology}
\label{subsec:protein_cc_go_top}

\todo{} Brief description of topology and possible figures.

\warning{} This line is a warning!

\note{} And this is an important note!

\todo{} Mass table.


\subsection{Potentials}
\label{subsec:protein_cc_go_potential}

\todo{} Potential components...

\subsubsection{Bond}
\label{sec:protein_cc_go_potential_bond}

\begin{smallpage}{Clementi G\={o} bond potential}<white>
  \begin{equation}
    \label{eq:protein_cc_go_local_bond}
    U_{bond} = \sum_{i}^{bonds} k_b (r_i - r_{i,0})^2 + 100 k_b (r_i - r_{i,0})^4.
  \end{equation}
  \tcblower
  \begin{itemize}
  \item Bonds involved:
    \begin{itemize}
    \item P-S
    \item S-B
    \item S-P
    \end{itemize}
  \item $r_{i, 0}$: bonds in the native structure of protein.
  \item $k_b = 0.6\ \mathrm{kJ/mol/\angstrom^2}$.
  \end{itemize}
\end{smallpage}



\subsubsection{Angle}
\label{sec:protein_cc_go_potential_angle}

\begin{smallpage}{Clementi G\={o} bond angle potential}<white>
  \begin{equation}
    \label{eq:protein_cc_go_local_angle}
    U_{ang} = \sum_{i}^{angles} k_\theta (\theta_i - \theta_{i,0})^2.
  \end{equation}
  \tcblower
  \begin{itemize}
  \item Angles involved:
    \begin{itemize}
    \item P-S-P
    \item S-P-S
    \item P-S-B
    \item B-S-P
    \end{itemize}
  \item $\theta_{i, 0}$: based on the structure of B-form DNA.
  \item $k_\theta$: see the table below: (unit: $\mathrm{kJ/mol/rad^2}$)
  \end{itemize}
\end{smallpage}

