\chapter{Protein Models}
\label{chap:protein}


\section{Clementi G\={o}}
\label{sec:protein_cc_go}

Clementi \emph{et al.} proposed an off-lattice G\={o} model and successfully
applied it to small fast-folding proteins, getting consistent folding pathways
with experiments~\cite{Clementi2000}.  This model has been used by several
following works showing its capability of studying the protein folding reaction
mechanism~\cite{Hoang2000, Koga2001}.

\subsection{Topology}
\label{subsec:protein_cc_go_top}

In this minimal G\={o} model, each amino acid residue in a given protein is
represented by a single CG bead put at the C\textsubscript{$\alpha$} atom, with
unified mass $m_{i} = 1$.

\note{} In the protein G\={o} model simulations, people usually use the reduced
unites, \emph{i.e.}, $m=1, \epsilon=1, k_B=1$.  However, when the G\={o} model
is used for protein together with other CG biomolecular models such as the
3SPN.2C DNA, the user has to be careful about the units.

\subsection{Potentials}
\label{subsec:protein_cc_go_potential}

The Clementi G\={o} model includes several terms:
\begin{equation}
  \label{eq:protein_cc_go_all}
  U = U_{bond} + U_{ang} + U_{dih} + U_{nonlocal}
\end{equation}

\subsubsection{Bond}
\label{sec:protein_cc_go_potential_bond}

\begin{smallpage}{Clementi G\={o} bond potential}<white>
  \begin{equation}
    \label{eq:protein_cc_go_local_bond}
    U_{bond} = \sum_{i}^{bonds} k_b (r_i - r_{i,0})^2
  \end{equation}
  \tcblower
  \begin{itemize}
  \item Bonds involved:
    \begin{itemize}
    \item C\textsubscript{$\alpha$}-C\textsubscript{$\alpha$}
    \end{itemize}
  \item $r_{i, 0}$: $r_{i}$ in the native structure of protein.
  \item $k_b = 100.0$.
  \end{itemize}
\end{smallpage}


\subsubsection{Angle}
\label{sec:protein_cc_go_potential_angle}

\begin{smallpage}{Clementi G\={o} bond angle potential}<white>
  \begin{equation}
    \label{eq:protein_cc_go_local_angle}
    U_{ang} = \sum_{i}^{angles} k_\theta (\theta_i - \theta_{i,0})^2.
  \end{equation}
  \tcblower
  \begin{itemize}
  \item Angles involved:
    \begin{itemize}
    \item C\textsubscript{$\alpha$}-C\textsubscript{$\alpha$}-C\textsubscript{$\alpha$}
    \end{itemize}
  \item $\theta_{i, 0}$: $\theta_{i}$ in the native structure of protei.
  \item $k_\theta = 20.0$.
  \end{itemize}
\end{smallpage}


\subsubsection{Dihedral Angle}
\label{sec:protein_cc_go_potential_dihedral_angle}

\begin{smallpage}{Clementi G\={o} dihedral angle potential}<white>
  \begin{equation}
    \label{eq:protein_cc_go_local_dihedral_angle}
    U_{dih} = \sum_{i}^{dihedrals} k_{\phi}^{(n)} \big[ 1+\cos\big(n( \phi_i - \phi_{ i,0 } )\big) \big].
  \end{equation}
  \tcblower
  \begin{itemize}
  \item Angles involved:
    \begin{itemize}
    \item C\textsubscript{$\alpha$}-C\textsubscript{$\alpha$}-C\textsubscript{$\alpha$}-C\textsubscript{$\alpha$}
    \end{itemize}
  \item $\phi_{i, 0}$: native value shifted by $\pi/n$.
  \item $k_\phi^{(1)} = 1.0$.
  \item $k_\phi^{(3)} = 0.5$.
  \end{itemize}
\end{smallpage}

\note{} The dihedral angle potential in GENESIS has the form of $k_{\phi} \big[
1+\cos(n\phi - \delta)\big]$, where we should set $\delta = n\phi_0 + \pi$.


\subsubsection{Nonlocal Interactions}
\label{sec:protein_cc_go_potential_nonlocal}

\begin{smallpage}{Clementi G\={o} native contact potential}<white>
  \begin{equation}
    \label{eq:protein_cc_go_native_contact}
    U_{contact} = \sum_{i}^{native contact} \varepsilon^{(n)} \big[ 5 \big( \frac{\sigma_{ij}}{r_{ij}} \big)^{12} - 6 \big( \frac{\sigma_{ij}}{r_{ij}} \big)^{10}\big].
  \end{equation}
  \tcblower
  \begin{itemize}
  \item Angles involved:
    \begin{itemize}
    \item C\textsubscript{$\alpha$}-C\textsubscript{$\alpha$}-C\textsubscript{$\alpha$}-C\textsubscript{$\alpha$}
    \end{itemize}
  \item $\phi_{i, 0}$: native value shifted by $\pi/n$.
  \item $k_\phi^{(1)} = 1.0$.
  \item $k_\phi^{(3)} = 0.5$.
  \end{itemize}
\end{smallpage}

